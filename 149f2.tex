\section{Introduction}
Preventative approaches for cardiovascular disease have had a profound effect on reducing adult morbidity and mortality in the 20th century. Despite this, cardiovascular disease will remain one of the most important health challenges in the 21st century. Prolonged exposure to lifestyle and other types of risk factors make meaningful change in disease risk difficult to achieve as adults, particularly after the successes in cardiovascular mortality reduction since the 1970’s.
New goals such as the World Heart Federation’s ‘25 by 25’ campaign1 or the American Heart Association’s 2020 Impact Goal2 represent a fundamental shift in global emphasis to primordial prevention and supporting the attainment of ideal health rather than treatment of disease risk.
Atherosclerosis, the main pathological mechanism in cardiovascular disease, begins early in life. Compelling evidence for this already exists in autopsy and imaging studies. It is also clear that exposure to risk factors and behaviours such as dietary habits start in childhood and adolescence, setting individuals on differing health trajectories later in life. 
An understanding of early life risk factors, and their immediate effects on atherosclerosis, provides insights into the natural origins of cardiovascular disease and the potential effectiveness and appropriate timing of interventions. Should the life-course model of health and disease prove informative, earlier intervention, during periods of childhood plasticity, will have lasting effects later in life. To realise these insights, investigation of traditional and novel risk factors, and their relation to cardiovascular health trajectories, is required. 
